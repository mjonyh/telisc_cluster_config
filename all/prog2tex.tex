\documentclass[twoside,12pt]{article}

\usepackage{prog2tex}
\usepackage[paper=a4paper,dvips,top=1.5cm,left=1.5cm,right=1.5cm,foot=1cm,bottom=1.5cm]{geometry}

\newcommand{\bs}{\textbackslash}

\title{\progtex{} v2.0 Documentation}
\author{Tal Cohen}

\begin{document}
\maketitle

\begin{abstract}
\progtex{} is a software tool that allows the inclusion of C++, Java and BNF (Jamoos)
    programs in \LaTeX{} documents.
Short program excerpts, whole program files or selected ranges from program files can
    all be added, as well as small program expressions as part of the flowing text.
\end{abstract}

\section{Introduction}

This is a brief documentation of the \progtex{} utility, as well
    being a test document (as it uses practically all of
    \progtex{}'s features).
Also included is a technical discussion of the changes made
    since the program's last version.

\progtex{} allows the inclusion of C++, Java, and BNF (Jamoos) programs in
    \LaTeX{} documents.
Source code written in any of these languages can be
    included in one of four ways:

\begin{enumerate}

\item{\textbf{Code paragraphs}} inside the \LaTeX{} document. This method can
    be used to include blocks of source code directly inside the
    \LaTeX{} document.

\item{\textbf{Inlined code}} inside the \LaTeX{} text, useful for quoting a small
    fragment of code (e.g., a single statement or expression) without breaking
    the flow of reading.

\item{\textbf{Included files}} allows the inclusion of source code directly from
    source files (e.g., \texttt{.java}, \texttt{.h} or \texttt{.cpp}
    files).

\item{\textbf{Included file range}} enables the inclusion of only a limited part (a
    \emph{named range}) from a source file.
The named ranges must be defined in the source file itself (using special
    delimiters that would seem like comments to the compiler).

\end{enumerate}

\paragraph{Outline} The next section explains how \progtex{} should be run.
Sections~\ref{Section:Code:Paragraphs} to~\ref{Section:Included:File:Ranges}
    detail the exact syntax for each of the four source-inclusion modes.
Section~\ref{Section:Including:Latex:Files} deals with using \LaTeX{}'s normal
    file inclusion commands (\texttt{\bs{}include} and \texttt{\bs{}input}).
Section~\ref{Section:Using:Latex:Notation} describes how \LaTeX{} formatting
    commands can be added to comments in source code.
Both Section~\ref{Section:Technical:Notes}, which contains technical notes, and
    Section~\ref{Section:Future:Changes}, are aimed for future maintainers of
    \progtex{}, and would be of no interest to most users.
Finally, Section~\ref{Section:Conclusion} is a brief conclusion.

\section{Using \progtex{}}

To include source code in \LaTeX{} documents, the following steps should
    be taken:

\begin{enumerate}

\item Add \texttt{\bs{}usepackage\{prog2tex\}} to the document's
    preamble.

\item Use \progtex{}'s commands in your document.
\progtex{}'s commands include the various source-inclusion macros (detailed in the
    following sections), as well as the
    \texttt{\bs{}progtex} macro, used to generate
    \progtex{}'s ``logo''.

\item Before running \texttt{latex2e} on the document, run
    \texttt{prog2tex}.
This is done using the syntax ``\texttt{prog2tex} \textit{filename}''. If
    the \textit{filename} does not include an extension, the default
    ``\texttt{.tex}'' extension is assumed.
\texttt{prog2tex} modifies the \texttt{.tex} file by adding special macros
    (``\texttt{\bs{}PROG\textit{xx}}'' macros) that can be
    ignored (or even deleted) when the document is edited again.
The program also creates an additional input file that the package uses (with a
    \texttt{.prg} extension).

\item You can now run \texttt{latex2e} normally on the document.

\end{enumerate}

If the source document is modified, \texttt{prog2tex} should be run again
    before each time \texttt{latex2e} is used.
Only one execution of \texttt{prog2tex} is needed every time the file is
    modified, even if you run \texttt{latex2e} several times.

By default, \texttt{prog2tex} updates the \texttt{.tex} file on which it
    operates.
If a second command-line argument is provided, \texttt{prog2tex} will generate a new
    output file (leaving the input file untouched). For example:
    ``\texttt{prog2tex} \textit{filename} \textit{outputfile}''.
Note, however, that if the input \texttt{.tex} file uses the \LaTeX{}
    macros \texttt{\bs{}input} or
    \texttt{\bs{}include}, the included files \emph{will}
    be updated even if a second argument was specified (see
    Section~\ref{Section:Including:Latex:Files} for details).

If \texttt{prog2tex} is run without any arguments, it will use \emph{stdin}
    for input and \emph{stdout} for output.
In addition, it will create a file called \texttt{stdin.prg} where the
    generated macros will be stored.

\section{Code Paragraphs}
\label{Section:Code:Paragraphs}

To enter source code directly inside your \LaTeX{} document, use the
    following syntax:

~

\texttt{\bs{}}\textit{lang}

\textit{source code}

\texttt{\bs{}END}

~

Where ``\texttt{\bs{}}\textit{lang}'' is one of ``\texttt{\bs{}BNF}'',
    ``\texttt{\bs{}CPP}'' or ``\texttt{\bs{}JAVA}'', for BNF,
    C/C++, or Java code, respectively.

For example, consider the following \LaTeX{} source:

\framebox{%
\begin{minipage}{0.9\textwidth}%
\texttt{%
\bs{}CPP \\%
// hello.c \\%
\\%
\#include $<$stdio.h$>$ \\%
\\%
int main(int argc, char \*argv[]) \\%
\{ \\%
\mbox{ }\mbox{ }\mbox{ }// just print\ldots \\%
\mbox{ }\mbox{ }\mbox{ }printf("Hello, World!\bs{}n"); \\%
\mbox{ }\mbox{ }\mbox{ }\\%
\mbox{ }\mbox{ }\mbox{ }return 0;\\%
\} \\%
\bs{}END
}%
\end{minipage}%
}

~

Once processed by \progtex{} and \LaTeX{}, it will produce the following
    result:

~

\framebox{%
\begin{minipage}{0.9\textwidth}%
\CPP
// hello.c

#include <stdio.h>

int main(int argc, char *argv[])
{
    // just print...
    printf("Hello, World!\n");

    return 0;
}
\END\PROGb{}
\end{minipage}
}

~

Note that keywords appear in \textbf{boldface}, and comments appear
    in a proportional font.
(The way various language parts are rendered can in fact be changed by
    updating the file \texttt{prog2tex.sty}, without otherwise
    modifying \progtex{}'s source code.)

\section{Inlined Code}

Sometimes, a small piece of code has to be included directly inside
    normal text.
\progtex{} supports this by using ``inline'' code macros.

The three inline macros are \texttt{\bs{}bnf}, \texttt{\bs{}cpp},
    and \texttt{\bs{}java}, for inlining BNF, C/C++, or Java code
    fragments, respectively.

Any text passed as a parameter to these macros will be processed to generate
    properly formatted source code (e.g., keywords will appear in bold, etc.).

For example, consider the following \LaTeX{} paragraph:

~

\framebox{%
\begin{minipage}{0.9\textwidth}%
\texttt{%
Constructors in Java must begin with calls to either \bs{}java\{this(\ldots)\}
    or \bs{}java\{super(\ldots)\}.
}%
\end{minipage}%
}

~

It will yield:

~

\framebox{%
\begin{minipage}{0.9\textwidth}%
Constructors in Java must begin with calls to either \java{this(...)}\PROGc{}
    or \java{super(...)}\PROGd{}.
\end{minipage}%
}

~

The characters `\texttt{\{}' and `\texttt{\}}' need not (and cannot)
    be escaped inside inlined macros.
This means that the text included inside the inline macro must not contain
    unbalanced curly braces.
For example, the following code will baffle \progtex, and cause unexpected
    results:

~

\framebox{%
\begin{minipage}{0.9\textwidth}%
\texttt{%
You must avoid bad usage of \bs{}java\{switch \{\} in your programs.
}%
\end{minipage}%
}

~

Inlined C/C++ code fragments can appear inside BNF code paragraphs,
    and vice versa.
The nesting can be repeated to any depth necessary.
For example, the following block of code is a BNF program that includes
    inlined C++ code, generated using ``\texttt{\bs{}cpp\{\ldots\}}'',
    inside a ``\texttt{\bs{}BNF\ldots{}\bs{}END}'' block:

~

\framebox{%
\begin{minipage}{0.9\textwidth}%
\BNF
    Main? -> argv: { STRING ... }+
    FEATURES
        p: PascalProgram? := PARSE(argv[\cpp{1}],PascalProgram);
        is_legal:OK := [[ \cpp{if (!\bnf{$p?}) \bnf{ERROR};} ]];
        triples:TriplesList := p.triples();
        optimized:TriplesList := triples.optimize();
        generate:OK := optimized.dump();
    END;
\END\PROGe{}
\end{minipage}%
}

~

(Note that the framebox is not automatically generated as well; it was included here,
    and in other examples, to increase this document's readability.)

\section{Included Files}

Not all source code has to reside inside the \LaTeX{} document.
Source files can be included directly, using the macros
    \texttt{\bs{}cppfile},
    \texttt{\bs{}bnffile} and
    \texttt{\bs{}javafile} (all-uppercase
    versions of these macros are also defined, e.g.,
    \texttt{\bs{}CPPFILE}).

These macros accept a single parameter, which is the filename to
    include.
If the file is not found, \progtex{} will try again, appending
    an extension to the name.
The extension used is \texttt{.cpp}, \texttt{.bnf} or
    \texttt{.java}, depending on the macro at hand.

As an example, here's the complete content of the file
    \texttt{sample.cpp}, included using the command
    ``\texttt{\bs{}cppfile\{sample\}}''. Again, the framebox
    was added here for increased readability.

~

\framebox{%
\begin{minipage}{0.9\textwidth}%
\cppfile{sample}\PROGf{}
\end{minipage}%
}

~

This file includes several named ranges, which will be used
    in the following section.

\section{Included File Ranges}
\label{Section:Included:File:Ranges}

Using \emph{named ranges}, \progtex{} supports the inclusion of
    selected parts from a source file (rather than including
    the entire file).

The beginning of a range inside a source file is marked using
    the string

\texttt{/*-\texttt{}-begin:}\textit{rangename}\texttt{-\texttt{}-*/}

The compiler will naturally consider this marker to be a comment, and
    ignore it.
In a similar manner, the end of a range is marked using the string

\texttt{/*-\texttt{}-end:}\textit{rangename}\texttt{-\texttt{}-*/}

To include a range, use the syntax

~

\framebox{%
\begin{minipage}{0.9\textwidth}%
\texttt{\bs{}cppfile\{}\textit{filename}\texttt{:}\textit{rangename}\texttt{\}}
\end{minipage}%
}

~

(The same can be done using \texttt{\bs{}javafile} and
    \texttt{\bs{}bnffile} as well.) For example, using the file \texttt{sample.cpp}
    from the previous Section, the command

~

\framebox{%
\begin{minipage}{0.9\textwidth}%
\texttt{\bs{}cppfile\{sample:inter\}}
\end{minipage}%
}

~

will yield this output:

~

\framebox{%
\begin{minipage}{0.9\textwidth}%
\cppfile{sample:inter}\PROGg{}
\end{minipage}%
}

~

Ranges can be nested, or even cross each other's borders.
In fact, a range does not even have to be continuous: if the same range
    name is used more than once, it is considered a single, non-continuous
    range.
For example, here is the range named ``\texttt{impl}'' from \texttt{sample.cpp}:

~

\framebox{%
\begin{minipage}{0.9\textwidth}%
\cppfile{sample:impl}\PROGh{}\PROGi{}
\end{minipage}%
}

~

The range markers themselves are not considered part of the range, and
    hence are not included in the \LaTeX{} document (though if a range
    $R_1$ contains the markers for some range $R_2$ internally,
    including range $R_1$ will show $R_2$'s markers, just as it would
    show any other comment contained within the range).

\section{Including \LaTeX{} Files}
\label{Section:Including:Latex:Files}

\LaTeX{} documents processed with \progtex{} can use the normal
    file-inclusion commands \texttt{\bs{}include} and \texttt{\bs{}input}.
In this case, \progtex{} will treat source code fragments found the included
    \LaTeX{} files as well.

For example, \emph{this very paragraph} appears in the file \texttt{included.tex}, and
	was included using a normal \texttt{\textbackslash{}input} command.

The file includes a simple Java program:

~

\framebox{%
\begin{minipage}{0.9\textwidth}%
\JAVA
public class Hello {
	/**
	 * main: the program's starting point.
	 * @param args the arguments passed on the command-line.
	 * @author every Jave programmer.
	 */
	public static void main(String[] args) {
		System.out.println("Hello World!");
	}
}
\END\PROGj{}
\end{minipage}%
}

~

Note how @-keywords in \texttt{javadoc} comments appear in boldface.

(The file \texttt{included.tex} ends here.)



The included files are processed by \progtex{} individually, and are modified
    if they include source fragments (code paragraphs, inlined code, included
    source files or included ranges). As with the main file, the modifications
    are limited to adding \texttt{\bs{}PROG\textit{xx}} macros that
    can be ignored when the file is edited.

Every run of \progtex{} generates a single macros (\texttt{.prg}) file, even if
    the main \texttt{.tex} file includes numerous source files and/or other
    \LaTeX{} files.

\section{Using \LaTeX{} Notation in Source Code}
\label{Section:Using:Latex:Notation}

When writing a program that will eventually be included in a \LaTeX{} document,
    you can include \LaTeX{} formatting commands in the program's comments.
This is done using ``\texttt{//\{\}}'' for single-line comments in C, C++ or Java,
    or ``\texttt{/*\{\}}'' for multi-line (block) comments (these are
    closed using the normal ``\texttt{*/}'' pair).

For example, consider the following \LaTeX{} code, containing a
    Java code fragment:

~

\framebox{%
\begin{minipage}{0.9\textwidth}%
\texttt{%
\bs{}JAVA \\%
\mbox{ }if (a[i] $>$ max) //\{\} i.e., \$a\_i\$ is the new maximum \\%
\mbox{ }\mbox{ }\mbox{ }\mbox{ }return Math.pow(x, Math.PI); //\{\} return \$x\^{}\bs{}pi\$ \\%
\bs{}END %
}%
\end{minipage}%
}

~

And its result:

~

\framebox{%
\begin{minipage}{0.9\textwidth}%
\JAVA
 if (a[i] > max) //{} i.e., $a_i$ is the new maximum
    return Math.pow(x, Math.PI); //{} return $x^\pi$
\END\PROGba{}
\end{minipage}%
}

~

All one-line comments (``\texttt{-\texttt{}-}'') in BNF code automatically
    support \LaTeX{} formatting commands.

\texttt{javadoc} comments cannot include \LaTeX{} formatting.

\section{Technical Notes}
\label{Section:Technical:Notes}

This section provides a list of changes of note in \progtex{}'s source code
    (mainly the file \texttt{prog2tex.l}) since the previous version.
The previous version did not support inclusion (neither source code
    inclusion or the processing of included \LaTeX{} files), though
    the basic infrastructure for such support did exist.

\begin{enumerate}

\item
The code was heavily retouched for uniform indentation, and detailed comments
    was added.

\item
The \cpp{Buffer}\PROGbb{} construct had a problematic design and was improperly used.
    It was simply removed, since only a single output (\texttt{.prg}) file
    is generated anyway.

\item
Several elements (functions, flex start-conditions, etc.) were defined but never
    used. These include \cpp{BID}\PROGbc{}, \cpp{VAR}\PROGbd{}, \cpp{BVAR}\PROGbe{}, \cpp{Language::begin_block()}\PROGbf{},
    etc. They were all were removed.

\item
Several elements were renamed to more properly describe their usage (e.g., \cpp{d2a()}\PROGbg{} was
    renamed to \cpp{PROG_macro_id()}\PROGbh{}; \cpp{PASS}\PROGbi{} mode was renamed \cpp{LATEX}\PROGbj{};
    \cpp{CODE}\PROGca{} and \cpp{CODE1}\PROGcb{} were renamed \cpp{CPP}\PROGcc{} and \cpp{CPP1}\PROGcd{}; etc.).

\item
Since only a single macros file is generated, the file inclusion stack (\cpp{Includes}\PROGce{},
    but see below) no longer keeps track of the macro output file.

\item
Added support for a default extension to included source files.

\item
Some functions (e.g., \cpp{process()}\PROGcf{} and \cpp{sprocess()}\PROGcg{}) were changed so they now
    return a \cpp{bool}\PROGch{} value (instead of \cpp{int}\PROGci{}).

\item
Several global status variables (e.g., \cpp{line}\PROGcj{}, \cpp{pos}\PROGda{}, etc.) are now stored
    per source file. To this end, they are now defined as fields in the
    \cpp{FileState}\PROGdb{} class.

\item
The file inclusion stack (\cpp{Includes}\PROGdc{}) is now used for both \LaTeX{} and included
    source files. This allows proper handling of both source code includes and
    \LaTeX{} \texttt{\bs{}include} and \texttt{\bs{}insert}
    commands. Even the main file is considered ``included'', and is
    pushed onto this stack at program start. Hence, the class was renamed
    \cpp{Sources}\PROGdd{} (since it is used even if not a single file is
    \texttt{\bs{}include}'d).

\end{enumerate}

Several bugs in the original code were found and corrected.
This includes ``active'' bugs, as well as ``dormant'' bugs in the incomplete
    infrastructure for supporting file inclusion.

Some (fixed) bugs of interest include:

\begin{enumerate}

\item
The \cpp{main()}\PROGde{} function relied on a random non-zero initialization of a local
    variable to work properly.
That variable was actually not needed at all.

\item
If the processed \texttt{.tex} file was not in Unix's \texttt{pwd} (present working
    directory), the temp file was generated in \texttt{pwd} but never deleted.
Now the temp file is always created in the \texttt{.tex} file's directory, and
    deleted (except in cases of abnormal termination).

\item
The initial state was reported (by the mode-tracking debug facilities) as \cpp{INITIAL}\PROGdf{},
    regardless of any calls to \cpp{BEGIN}\PROGdg{} before \cpp{yylex()}\PROGdh{} starts.
This was due to an improper initialization of the variable \cpp{state}\PROGdi{}, plus an
    ``unintercepted'' call to \cpp{BEGIN}\PROGdj{} in \cpp{process()}\PROGea{}.

\item
The global variable \cpp{lang}\PROGeb{} (of type \cpp{Language *}\PROGec{}) was constantly assigned new
    instances of \cpp{Cpp}\PROGed{}/\cpp{Bnf}\PROGee{}, with no cleanup.
To fix this problem, three ``singleton''\footnote{These are simply three global
variables, one per class. They do not actually follow the \textsc{Singleton} design pattern,
even though such a change is possible.} instances of the \cpp{Cpp}\PROGef{}, \cpp{Bnf}\PROGeg{}, \cpp{Java}\PROGeh{}
    classes are used.
\cpp{lang}\PROGei{} is used to point to relevant instance instead of a new one each time.
This is possible since the only member field of \cpp{Language}\PROGej{} is \cpp{name}\PROGfa{}, which does not
    change during the object's lifecycle.

\item
If an included file ended abruptly (e.g., inside a comment), the parser would lose sync
    and the state-stack would become meaningless.

\item
The line number report in the generated \texttt{.prg} file (stating from which line in
    the file was the code fragment taken) generated incorrect numbering, and could not
    support multiple input files.
This was fixed, and the report format changed from ``\texttt{Line }\textit{n}'' to
    ``\textit{filename}\texttt{:}\textit{n}'' (e.g., ``\texttt{test.tex:15}'').

\item
LaTeX-style support for block comments was not working, since whomever added the
    \cpp{ALIGN}\PROGfb{} mode did it in a way that broke that support (specifically, when
    returning from \cpp{ALIGN}\PROGfc{}, the code always returned to \cpp{COMMENT}\PROGfd{} mode,
    never to \cpp{LATEX_COMMENT}\PROGfe{}).

\item
The \cpp{ALIGN}\PROGff{} mode caused loss of tab characters in the source file.

\end{enumerate}

\section{Future Changes}
\label{Section:Future:Changes}

``Plan to throw one away; you will, anyhow'' (Fredrick P. Brooks, Jr., \emph{The Mythical
    Man-Month}).

It is time to ``throw one away'' for this program; the current design has outlived its
    usefulness, and if another major upgrade is needed, the best way to do it would be
    a complete rewrite (in a programming language worthy of the name, if possible; in
    other words, not in C/C++).

Other than a complete rewrite, here are a few suggestions for improving the existing design:

\begin{enumerate}

\item
The abstract \cpp{Language}\PROGfg{} class and its concrete language-specific subclasses is a good
    idea, that came into use too late.
Currently, it is used only for included files.
However, there's no reason not to use it for any code fragment, of any type.

\item
The \cpp{Language}\PROGfh{} class can be put to additional uses other than those currently employed
    (which are mainly opening and closing blocks in a language-specific manner).
In particular, it can be used for detecting language keywords.
This would allow adding support for new languages with significantly less hassle, since
    the flex grammar need not be updated for the keywords of each new language.
Instead, knowledge about the keywords will simply be stored in the new subclass of
    \cpp{Language}\PROGfi{}.

\item
File inclusion is currently handled as a stack (the \cpp{Sources}\PROGfj{} variable, previously
    called \cpp{Includes}\PROGga{}).
It would be better to change this into a queue, with a mechanism that would prevent
    duplicate entries.
This change would have two direct advantages:
(a) if a source file is included more than once, it will only be processed once, and
(b) in the unlikely case that two \LaTeX{} files mutually include each other (technically
    possible, since conditional \LaTeX{} commands can be used to prevent endless nesting),
    the current design would reach an endless inclusion loop, whereas a
    queue design would not.

\item
\progtex{} macros found inside \LaTeX{} comments are currently processed just like any
    other.
The program can be taught about \LaTeX{} comments so it would ignore them.

\end{enumerate}

\section{Conclusion}
\label{Section:Conclusion}

It still surprises me, considering their origin and authors, that while both
    \TeX{} abd \LaTeX{} include built-in support for poetry and verse, neither includes
    proper handling of source code inside documents.



\end{document}
