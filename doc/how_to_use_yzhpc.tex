\documentclass{book}
\usepackage[dvipsnames]{xcolor}
\title{Welcome to YZHPC \\ 
{\small High Performance Computing Laboratory \\
Department of Physics \\
Shahjalal University of Science and Technology \\
Sylhet - 3114, Bangladesh}
}

%\author{Dr. Md. Enamul Hoque}
\date{Updated on \today}

\begin{document}
\maketitle
\newpage


%% Introduction chapter
\chapter{Introduction}%
\label{cha:introduction}

Configuration of the HPC:
\begin{description}
	\item[Hardware] Available hardware
		\begin{description}
			\item[Compute Node: ] 05
				\begin{itemize}
					\item CPU: Intel Xeon 4-core 2.1 GHz
					\item RAM: 8 GB 800 MHz
				\end{itemize}
			\item[Compute Node: ] 16
				\begin{itemize}
					\item CPU: Intel Core 2 Duo 2.6 GHz
					\item RAM: 4 GB 800 MHz
				\end{itemize}
			\item[Total Core: ] 52 Compute Core
			\item[Total RAM] 104 GB
		\end{description}
	\item[Software] Available Software.
		\begin{itemize}
			\item OS: Telisc OS
			\item Module Environment: lmod
			\item Task/Job Management and Schedular: SLURM
		\end{itemize}
	\item[Available Computing Packages] OpenMPI, LAMMPS
	\item[Funded by] Castle.com (Sylhet, BD), Prof. Md. Zafar Iqbal, HEQEP (CP - 4044), SUST Research Center
\end{description}

%% End of Introduction chapter

\chapter{Using shell in YZHPC}%
\label{cha:using_shell_in_yzhpc}

\section{Login}%
\label{sec:login}

Only SSH access is available to login in the system. Any SSH client from various Operating System can be used. \textcolor{red}{Additionally a web browser can be used to get login (firefox, google-chrome, Internet Explorer and Microsft edges were tested)}.

\subsection{From Linux or Mac Computer}%
\label{sub:from_linux_or_mac}

\begin{description}
	\item[Login] Open a terminal that available in your system. After that use \texttt{\$ ssh username@yzhpc.sust.edu} to login in the HPC head node.
	\item[Logout] Execute the \texttt{\$ exit} command.
\end{description}

\subsection{From Windows Computer}%
\label{sub:from_windows_computer}

A number of ssh client is available for windows machine. PuTTY is one of the best ssh client for windows machine.

\begin{enumerate}
	\item Open PuTTY from start menu.
	\item Provide username and address (i.e. yzhpc.sust.edu).
	\item Login to the system.
	\item Execute exit command to logout.
\end{enumerate}

\section{Useful Linux Command}%
\label{sec:useful_linux_command}

Use man COMMAND to do more with the following commands:

\begin{description}
	\item[ls] Very useful commands for documents listing in the current directory.
	\item[cd] To jump into another directory.
	\item[mkdir] Use it for creating new directory.
	\item[rm] To delete document it is used.
	\item[mv] To rename document, it is used.
	\item[cp] To copy document, it is used.
	\item[man] To view the manual of a package, it is used.
	\item[tail] Very useful command for viewing live output.
	\item[head] It shows the top few lines of a document.
	\item[less] It shows the document in visual mode.
	\item[cat] Shows all the content of a document.

\end{description}

\section{Useful HPC command}%
\label{sec:useful_hpc_command}

\begin{description}
	\item[sinfo] To view the status of available nodes.
	\item[squeue] To view the status of the jobs.
	\item[sacct] To view the details of the jobs.
	\item[module] To find the available software installed in the system.
\end{description}

\section{File transfer}%
\label{sec:file_transfer}

\subsection{Using rsync}%
\label{sub:using_rsync}

\subsection{Using scp}%
\label{sub:using_scp}



%% End of Using shell in YZHPC chapter


\chapter{Conclusion}%
\label{cha:conclusion}

This is the Conclusion.

\end{document}
