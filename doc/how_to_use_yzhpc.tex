\documentclass{book}
\title{Welcome to YZHPC \\ 
{\small High Performance Computing Laboratory \\
Department of Physics \\
Shahjalal University of Science and Technology \\
Sylhet - 3114, Bangladesh}
}

%\author{Dr. Md. Enamul Hoque}
\date{Updated on 2019-08-06}

\begin{document}
\maketitle
\newpage


%% Introduction chapter
\chapter{Introduction}%
\label{cha:introduction}

This is the Introduction.

%% End of Introduction chapter

\chapter{Using shell in YZHPC}%
\label{cha:using_shell_in_yzhpc}

\section{Login}%
\label{sec:login}

\section{Useful Linux Command}%
\label{sec:useful_linux_command}

Use man COMMAND to do more with the following commands:

\begin{description}
	\item[ls] Very useful commands for documents listing in the current directory.
	\item[cd] To jump into another directory.
	\item[mkdir] Use it for creating new directory.
	\item[rm] To delete document it is used.
	\item[mv] To rename document, it is used.
	\item[cp] To copy document, it is used.
	\item[man] To view the manual of a package, it is used.
	\item[tail] Very useful command for viewing live output.
	\item[head] It shows the top few lines of a document.
	\item[less] It shows the document in visual mode.
	\item[cat] Shows all the content of a document.

\end{description}

\section{Useful HPC command}%
\label{sec:useful_hpc_command}

\begin{description}
	\item[sinfo] To view the status of available nodes.
	\item[squeue] To view the status of the jobs.
	\item[sacct] to view the details of the jobs.
\end{description}

\section{File transfer}%
\label{sec:file_transfer}

\subsection{Using rsync}%
\label{sub:using_rsync}

\subsection{Using scp}%
\label{sub:using_scp}



%% End of Using shell in YZHPC chapter


\chapter{Conclusion}%
\label{cha:conclusion}

This is the Conclusion.

\end{document}
